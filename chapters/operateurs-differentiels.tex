% !TEX encoding = UTF-8 Unicode
\setchapterstyle{kao}
\setchapterpreamble[u]{\margintoc} 
\chapter{OPÉRATEURS DIFFÉRENTIELS}
\labch{operateurs_differentiels}

Complément de cours sur ce qu'il faut savoir à propos des opérateurs différentiels utilisés en physique.

\begin{center}
\textbf{Version en ligne}

\url{https://femto-physique.fr/omp/operateurs-differentiels.php}
\end{center}


% ------ section I ------
\section{L'opérateur gradient}
\subsection{Définition}

L'opérateur gradient est un \emph{opérateur différentiel} qui s'applique à un \textbf{champ scalaire} (fonction scalaire dépendant de l'espace et du temps) et le transforme en un \textbf{champ vectoriel} (vecteur dépendant de l'espace et du temps). Il se lit « gradient » ou « nabla » et se note : 
\[
	\overrightarrow{\text{grad}}f(\text{M},t)\quad\text{ou}\quad\overrightarrow{\nabla}f(\text{M},t)
\]
Dans le système de cordonnées cartésiennes le gradient s'exprime ainsi: 
% --- equation --- (fold)
\begin{equation}
\fcolorbox{filet}{fond}{\hspace{0.5em}
\(\displaystyle 
\overrightarrow{\text{grad}}f(x,y,z,t) = 
\dfrac{\partial f(x,y,z,t)}{\partial x}\overrightarrow{u_{x}} + \dfrac{\partial f(x,y,z,t)}{\partial y}\overrightarrow{u_{y}} + \dfrac{\partial f(x,y,z,t)}{\partial z}\overrightarrow{u_{z}}
\)\hspace{0.5em}}
\hspace{0.5em}\heartsuit
\end{equation}
%--- equation --- (end)
La \reftab{expressions_de_l_operateur_gradient_dans_differents_systemes_de_coordonnees}  donne les différentes expressions du gradient dans les systèmes de coordonnées utilisés couramment en physique.
\begin{table}
\centering
\footnotesize
\begin{tabular}{c|c|l}
\toprule
\textbf{Système}		&	$f(\text{M},t)$				& \textbf{Expression de} $\text{grad}f$					  \\[4mm]
Cartésien											&	$f(x,y,z,t)$					& $\dfrac{\partial f}{\partial x}\overrightarrow{u_{x}} + \dfrac{\partial f}{\partial y}\overrightarrow{u_{y}} + \dfrac{\partial f}{\partial z}\overrightarrow{u_{z}}$ \\[4mm]										
Cylindriques & $f(r,\theta,z,t)$  & $\dfrac{\partial f}{\partial r}\overrightarrow{u_{r}}+\dfrac{\partial f}{r\partial\theta}\overrightarrow{u_{\theta}}+\dfrac{\partial f}{\partial z}\overrightarrow{u_{z}}$\\[4mm]
Sphériques   & $f(r,\theta,\varphi,t)$ & $\dfrac{\partial f}{\partial r}\overrightarrow{u_{r}}+\dfrac{\partial f}{rd\theta}\overrightarrow{u_{\theta}}+\dfrac{\partial f}{r\sin\theta d\varphi}\overrightarrow{u_{\varphi}}$\\
\bottomrule
\end{tabular}
\caption{Expressions de l'opérateur gradient dans différents systèmes de coordonnées.}
\labtab{expressions_de_l_operateur_gradient_dans_differents_systemes_de_coordonnees}
\end{table}


\exercice{Calculer le gradient des champs suivants : \(f(x,y,z)=\dfrac{1}{2}(x^{2}+y^{2}+z^{2})\) et \(g(r,\theta,\varphi)=-\frac{1}{r}\).\\[2mm]
\emph{Rép.} \(\overrightarrow{\nabla}f=(x,y,z)=\overrightarrow{\text{OM}}\) et \(\overrightarrow{\nabla}g=\frac{1}{r^2}\overrightarrow{u_r}\).}
\subsection{Propriétés}
L'opérateur gradient est un  opérateur linéaire et vérifie donc 
\[
	\overrightarrow{\nabla}(\alpha f+\beta g)=\alpha \overrightarrow{\nabla}f+\beta\overrightarrow{\nabla}g
	\quad\text{avec}\quad	(\alpha,\beta)\in\mathbb{R}^2
\]
Le gradient d'un produit de champs scalaires vaut 
\[
	\overrightarrow{\nabla}(f.g)=f\overrightarrow{\nabla}g+g\overrightarrow{\nabla}f	
\]
où $f$ et $g$ sont deux fonctions de l'espace et du temps.

\textbf{Lien avec la différentielle --} On peut définir le gradient à partir de sa relation avec la différentielle. Soit M un point de l'espace et M' un point infiniment voisin, la différentielle $\text{d}f$ représente la variation du champ scalaire $f$ lorsque l'on se déplace de M à M' à $t$ fixé : 
	\[
	\text{d}f\stackrel{\text{def}}= f(\text{M'},t)-f(\text{M},t)=\overrightarrow{\nabla}f(\text{M},t)\cdot\overrightarrow{\text{d}\ell}
	\quad\text{avec}\quad
	\overrightarrow{\text{d}\ell}=\overrightarrow{\text{MM'}}	
	\]
En conséquence,
\begin{itemize}
\item Le vecteur $\overrightarrow{\nabla}f(\text{M},t)$ est perpendiculaire à la surface de niveau\sidenote{La surface de niveau de $f$ est l'ensemble des points M pour lesquels $f(M,t)$ conserve la même valeur à un instant $t$ fixé. En dimension $d=2$, cet ensemble donne une courbe de niveau.} de $f$ passant par M à l'instant $t$.
\item Le vecteur gradient est orienté vers les valeurs croissantes de $f$ et sa norme  mesure le taux de variation spatiale dans la direction de plus grande pente 
\[
	\left\| \overrightarrow{\nabla}f\right\| =\dfrac{\text{d}f}{\text{d}\ell}
\]
\end{itemize}

\exercice{Considérons le champ scalaire de l'espace bi-dimensionnel,  \(f(x,y)=x^2+y^2\). Représenter les courbes de niveau puis calculer \(\overrightarrow{\nabla}f\). Tracer quelques vecteurs gradients.\\[2mm]
\emph{Rép.} Les courbes de niveau sont des cercles de centre O. On a \(\overrightarrow{\nabla}f=(2x,2y)=2\overrightarrow{\text{OM}}\). Les vecteur gradients sont effectivement perpendiculaires aux cercles.}



% ------ section II ------

\section{L'opérateur divergence}
\subsection{Définition}
L'opérateur divergence est un opérateur différentiel qui s'applique à un \textbf{champ vectoriel} et qui renvoie un \textbf{champ scalaire}. Il se lit « divergence » et se note : 
	\[
\text{div}\overrightarrow{A}(\text{M},t)\quad\text{ou}\quad\overrightarrow{\nabla}\cdot\overrightarrow{A}(\text{M},t)
	\]
Cette notation permet de retenir l'expression de la divergence en coordonnées cartésiennes : 
% --- equation --- (fold)
\begin{equation}
\fcolorbox{filet}{fond}{\hspace{0.5em}
\(\displaystyle 
\text{div}\overrightarrow{A}(x,y,z,t)=
\left(\begin{array}{c}
\partial/\partial x\\
\partial/\partial y\\
\partial/\partial z
\end{array}\right)
\cdot
\left(\begin{array}{c}
A_{x}\\
A_{y}\\
A_{z}\\
\end{array}\right) = \left.\frac{\partial A_x}{\partial x}\right|_{y,z}
+ \left.\frac{\partial A_y}{\partial y}\right|_{x,z}
+\left.\frac{\partial A_z}{\partial z}\right|_{x,y}
\)\hspace{0.5em}}
\hspace{0.5em}\heartsuit
\end{equation}
%--- equation --- (end)
La \reftab{expressions_de_la_divergence_dans_differents_systemes_de_coordonnees} donne les différentes expressions de la divergence d'un champ vectoriel exprimé dans différents systèmes de coordonnées.
\begin{table}[htbp]
\centering
\footnotesize
\begin{tabular}{c|l}
	\toprule
	\textbf{Système}	& Expression de $\text{div}\overrightarrow{A}=\nabla\cdot\overrightarrow{A}$\\[4mm]
	cartésien		& $ \left.\frac{\partial A_x}{\partial x}\right|_{y,z}
+ \left.\frac{\partial A_y}{\partial y}\right|_{x,z}
+\left.\frac{\partial A_z}{\partial z}\right|_{x,y}$\\[4mm]
	cylindriques		& $\dfrac{\partial(r A_{r})}{r\partial r}+ \dfrac{\partial(A_{\theta})}{r\partial \theta} + \dfrac{\partial A_{z}}{\partial z}$\\[4mm]
	sphériques			& $\dfrac{1}{r^{2}}\dfrac{\partial(r^{2}\,A_{r})}{\partial r} +  \dfrac{1}{r\sin\theta}\dfrac{\partial(\sin\theta\,A_{\theta})}{\partial \theta} + \dfrac{1}{r\sin\theta}\dfrac{\partial A_{\varphi}}{\partial \varphi}$ \\
	\bottomrule
\end{tabular}
\caption{Expressions de la divergence dans différents systèmes de coordonnées.}
\labtab{expressions_de_la_divergence_dans_differents_systemes_de_coordonnees}
\end{table}

\exercice{Considérons le champ vectoriel $\overrightarrow{A}(r,\theta,\varphi)=\dfrac{\overrightarrow{u_r}}{r^2}$. Calculer la divergence de ce champ en tout point M autre que O.\\[2mm]
\emph{Rép.} On trouve \(\text{div}\overrightarrow{A}=0\). On dit que \(\overrightarrow{A}\) est un champ à flux conservatif (sauf en O).}
\subsection{Propriétés}
L'opérateur divergence est un  \textbf{opérateur linéaire} et vérifie donc 
\[
	\text{div}(\alpha \overrightarrow{A}+\beta \overrightarrow{B})=\alpha\,\text{div}\overrightarrow{A}+
	\beta\,\text{div}\overrightarrow{B}
	\quad\text{avec}\quad	(\alpha,\beta)\in\mathbb{R}^2
\]
La divergence d'un produit vaut
\[
	\text{div}(f.\overrightarrow{A}) = \overrightarrow{\nabla}\cdot(f\overrightarrow{A}) = f\overrightarrow{\nabla}\cdot\overrightarrow{A}+\overrightarrow{A}\cdot\overrightarrow{\nabla}f = f\text{div}\overrightarrow{A}+\overrightarrow{A}\cdot\overrightarrow{\text{grad}}f
\]
La divergence d'un champ est reliée au calcul du flux.
\begin{kaobox}[frametitle=Théorème de Green-Ostrogradsky ou théorème de la divergence]
Le flux d'un champ vectoriel $\overrightarrow{A}(\textrm{M})$ à travers une surface fermée $(S)$ est égal à l'intégrale sur le volume $V$ limité par $(S)$ de la divergence du champ vectoriel. 
\[
\iint_{\textrm{M}\in (S)}\overrightarrow{A}(\textrm{M})\cdot\overrightarrow{\textrm{d}S}^{\textrm{ext}}=
\iiint_{\textrm{M}\in V}\text{div}\overrightarrow{A}(\textrm{M})\;\text{d}\tau
\quad\textrm{avec}\quad
\textrm{div}\overrightarrow{A}=\overrightarrow{\nabla}\cdot\overrightarrow{A}\]
\end{kaobox} 

\textbf{Sens physique --} La divergence prend un sens bien précis en mécanique des fluides. Considérons une portion de fluide en mouvement dans un fluide décrit par le champ de vitesse $\overrightarrow{v}(\text{M},t)$. Au cours du mouvement, le volume $\mathcal{V}$ de cette portion varie suite aux déformations engendrées par l'écoulement. La divergence de la vitesse est liée au  taux de dilatation de la portion fluide par la relation 
\[
	\text{div}\overrightarrow{v}=\frac{1}{\mathcal{V}}\frac{\text{D}\mathcal{V}}{\text{D}t}
\]


\section{L'opérateur rotationnel}
\subsection{Définition}
L'opérateur rotationnel est un opérateur différentiel qui transforme un \textbf{champ vectoriel} en un autre \textbf{champ vectoriel}. Il se lit « rotationnel » et se note 
	\[
	\overrightarrow{\text{rot}}\,\overrightarrow{A}(\text{M},t)
	\quad\text{ou}\quad
	\overrightarrow{\nabla}\wedge\overrightarrow{A}(\text{M},t)
	\]
Cette notation permet de retenir l'expression du rotationnel en coordonnées cartésiennes : 
% --- equation --- (fold)
\begin{equation}
\fcolorbox{filet}{fond}{\hspace{0.5em}
\(\displaystyle 
\overrightarrow{\text{rot}}\,\overrightarrow{A}=
\left(\begin{array}{c}
\dfrac{\partial}{\partial x}\\[4mm]
\dfrac{\partial}{\partial y}\\[4mm]
\dfrac{\partial}{\partial z}
\end{array}
\right)\wedge\left(
\begin{array}{c}
	A_{x}\\[5mm]
	A_{y}\\[5mm]
	A_{z}\\[5mm]
\end{array}
\right)=\left(
\begin{array}{c}
	\dfrac{\partial A_{z}}{\partial y}-\dfrac{\partial A_{y}}{\partial z}\\[4mm]
	\dfrac{\partial A_{x}}{\partial z}-\dfrac{\partial A_{z}}{dx}\\[4mm]
	\dfrac{\partial A_{y}}{dx}-\dfrac{\partial A_{x}}{dy}
	\end{array}\right)
\)\hspace{0.5em}}
\hspace{0.5em}\heartsuit
\end{equation}
%--- equation --- (end)
La \reftab{expressions_du_rotationnel} donne les différentes expressions du rotationnel dans différents systèmes de coordonnées.

\begin{table*}
\centering
\footnotesize
\begin{tabular}{c|l}
\toprule
\textbf{Système} & $\overrightarrow{\text{rot}}\,\overrightarrow{A}=\overrightarrow{\nabla}\wedge\overrightarrow{A}$ \\
cartésien & $\left(\dfrac{\partial A_{z}}{\partial y}-\dfrac{\partial A_{y}}{\partial z},\,
\dfrac{\partial A_{x}}{\partial z}-\dfrac{\partial A_{z}}{\partial x},\,
\dfrac{\partial A_{y}}{\partial x}-\dfrac{\partial A_{x}}{\partial y}\right)$\\[4mm]
cylindrique & $\left(\dfrac{1}{r}\dfrac{\partial A_{z}}{\partial\theta}-\dfrac{\partial A_{\theta}}{\partial z},\,
\dfrac{\partial A_{r}}{\partial z}-\dfrac{\partial A_{z}}{\partial r},\,
\dfrac{1}{r}\dfrac{\partial(rA_{\theta})}{\partial r}-\dfrac{1}{r}\dfrac{\partial A_{r}}{\partial \theta}\right)$\\[4mm]
sphérique & $\left(\dfrac{1}{r\sin\theta}\dfrac{\partial(\sin\theta A_{\varphi})}{\partial\theta} - \dfrac{1}{r\sin\theta}\dfrac{\partial A_{\theta}}{\partial\varphi},\,
\dfrac{1}{r\sin\theta}\dfrac{\partial A_{r}}{\partial\varphi}-\dfrac{1}{r}\dfrac{\partial(rA_{\varphi})}{\partial r},\, \dfrac{1}{r}\dfrac{\partial(rA_{\theta})}{\partial r}-\dfrac{1}{r}\dfrac{\partial A_{r}}{d\theta}\right)$\\[4mm]
	\bottomrule
	\end{tabular}	
\caption{Expressions du rotationnel dans différents systèmes de coordonnées}
\labtab{expressions_du_rotationnel}
\end{table*}
\subsection{Propriétés}
L'opérateur rotationnel étant linéaire, on a 
\[
	\overrightarrow{\text{rot}}\left(\alpha \overrightarrow{A}+\beta \overrightarrow{B}\right) =
	\alpha\,\overrightarrow{\text{rot}}\overrightarrow{A}+\beta\,\overrightarrow{\text{rot}}\overrightarrow{B}		
	\quad\text{avec}\quad	(\alpha,\beta)\in\mathbb{R}^2
\]
Le rotationnel d'un gradient est nul.
\[
\overrightarrow{\text{rot}}\,\overrightarrow{\text{grad}}f = 
\overrightarrow{\nabla}\wedge(\overrightarrow{\nabla f})=\overrightarrow{0}
\]
La divergence d'un rotationnel est nulle.
\[
\text{div}\left(\overrightarrow{\text{rot}}\overrightarrow{A}\right)=
\overrightarrow{\nabla}\cdot\left(\overrightarrow{\nabla}\wedge \overrightarrow{A}\right)=0		
\]
Le rotationnel d'un produit vaut
\[
\overrightarrow{\text{rot}}\,f\overrightarrow{A}=\overrightarrow{\nabla}\wedge(f\overrightarrow{A})=
\overrightarrow{\nabla}f\wedge\overrightarrow{A}+f\overrightarrow{\nabla}\wedge\overrightarrow{A}=
\overrightarrow{\text{grad}}f\wedge\overrightarrow{A}+f.\overrightarrow{\text{rot}}\,\overrightarrow{A}
\]
Relation avec la circulation :
\begin{kaobox}[frametitle=Théorème de Stokes]
La circulation d'un champ vectoriel le long d'un contour $\mathcal{C}$ \textbf{fermé} et \textbf{orienté} est égal au flux du rotationnel de ce champ à travers une surface $\mathcal{S}$ délimité par $\mathcal{C}$.  
\[
	\oint_{\textrm{M}\in \mathcal{C}}\overrightarrow{A}(\textrm{M})\cdot\overrightarrow{\textrm{d}\ell}=
	\iint_{\textrm{M}\in \mathcal{S}}\overrightarrow{\text{rot}}\overrightarrow{A}(\textrm{M})\cdot
	\overrightarrow{\text{d}S}
\]
avec  $\overrightarrow{\text{d}S}$ orienté à partir du sens de parcours de $\mathcal{C}$ et de la règle du tire-bouchon.
\end{kaobox} 

\textbf{Sens physique --} En mécanique des fluides, le rotationnel du champ de vitesse d'un fluide en écoulement est lié à la vitesse de rotation  $\Omega$ des particules de fluide au cours de leur mouvement.
\[
\overrightarrow{\Omega}=\frac{1}{2}\overrightarrow{\textrm{rot}}\overrightarrow{v}
\]	 


\section{L'opérateur laplacien}\label{sec:operateur_laplacien}
\subsection{Le laplacien scalaire}
L'opérateur laplacien scalaire est un opérateur différentiel d'ordre deux qui transforme un champ scalaire en
un autre champ scalaire. Le laplacien scalaire s'obtient en prenant la divergence du gradient et se note 
$\triangle f(\text{M},t)$.
% --- equation --- (fold)
\begin{equation}
\fcolorbox{filet}{fond}{\hspace{0.5em}
\(\displaystyle 
	\triangle f(\text{M},t)=\text{div}(\overrightarrow{\text{grad}}f) =\nabla^{2}f=\dfrac{\partial^2 f}{\partial x^2}+\dfrac{\partial^2 f}{\partial y^2}+\dfrac{\partial^2 f}{\partial z^2}
\)\hspace{0.5em}}
\hspace{0.5em}\heartsuit
\end{equation}
%--- equation --- (end)


La \reftab{expressions_du_laplacien_dans_differents_systemes_de_coordonnees} donne les expressions du laplacien scalaire dans différents systèmes de coordonnées.
\begin{table}
\centering
\footnotesize
\begin{tabular}{c|l}
\toprule
\textbf{Système}		&	 \textbf{Expression de} $\triangle f$ \\[4mm]
cartésien & $\dfrac{\partial^{2}f}{\partial x^{2}}+\dfrac{\partial^{2}f}{\partial y^{2}}+\dfrac{\partial^{2}f}{\partial z^{2}}$\\[4mm]
cylindriques & $\dfrac{1}{r}\dfrac{\partial}{\partial r}\left(r\dfrac{\partial f}{\partial r}\right) + \dfrac{1}{r^2}\dfrac{\partial^{2}f}{\partial\theta^{2}}+\dfrac{\partial^{2}f}{\partial z^{2}}$\\[4mm]
sphériques & $\dfrac{1}{r^{2}}\dfrac{\partial}{\partial r}\left(r^{2}\dfrac{\partial f}{\partial r}\right) + \dfrac{1}{r^{2}\sin\theta}\dfrac{\partial}{\partial\theta}\left(\sin\theta\dfrac{\partial f}{\partial\theta}\right) + \dfrac{1}{r^{2}\sin^{2}\theta}\dfrac{\partial^{2}f}{\partial\varphi^{2}}$\\[4mm]
\bottomrule
\end{tabular}
\caption{Expressions du laplacien dans différents systèmes de coordonnées.}
\labtab{expressions_du_laplacien_dans_differents_systemes_de_coordonnees}
\end{table}
\subsection{Le laplacien vectoriel}
Le laplacien s'applique également à un champ vectoriel. Dans ce cas il renvoie un autre champ vectoriel et se note 
\[
	\triangle \overrightarrow{A}	
\]
Par définition, le laplacien vectoriel s'obtient à l'aide de l'identité 
\[
\overrightarrow{\text{rot}}\,\overrightarrow{\text{rot}}\overrightarrow{A} = 
\overrightarrow{\nabla}\wedge\left(\overrightarrow{\nabla}\wedge\overrightarrow{A}\right) = 
\overrightarrow{\nabla}\left(\overrightarrow{\nabla}\cdot\overrightarrow{A}\right) - 
\nabla^{2}\overrightarrow{A} = \overrightarrow{\text{grad}}(\text{div}\overrightarrow{A})-\triangle\overrightarrow{A}	
\]
En coordonnées cartésiennes, les vecteur unitaires étant fixes, le laplacien vectoriel d'un champ  $\overrightarrow{A}$ est tout simplement, un vecteur dont les composantes sont les laplaciens scalaires des composantes de $\overrightarrow{A}$ :
	\[
	\triangle\overrightarrow{A}(\text{M},t) = \left(\triangle A_{x}\right)\overrightarrow{u_{x}} + 	
	\left(\triangle A_{y}\right)\overrightarrow{u_{y}} + \left(\triangle A_{z}\right)\overrightarrow{u_{z}}
	\]

\section{Accélération d'une particule de fluide}
On a vu \sidecite{Roussel:2010} que l'accélération d'une particule de fluide située en M à l'instant $t$ pouvait s'obtenir à l'aide du champ de vitesse $\overrightarrow{v}(\text{M},t)$ :
\[
	\overrightarrow{a}(\text{M},t)=\dfrac{\partial \overrightarrow{v}}{\partial t} + \left(\overrightarrow{v}\cdot\overrightarrow{\nabla}\right)\overrightarrow{v}	
\]
où le dernier terme désigne la partie \textbf{convective} de l'accélération. Explicitons la composante suivant Ox de ce terme en utilisant l'égalité
$\overrightarrow{A}\wedge(\overrightarrow{B}\wedge\overrightarrow{C})=(\overrightarrow{A}.\overrightarrow{C})\overrightarrow{B}-(\overrightarrow{A}.\overrightarrow{B})\overrightarrow{C}$
avec $\overrightarrow{A}=\overrightarrow{v}$, $\overrightarrow{B}=\overrightarrow{\nabla}v_{x}$
et $\overrightarrow{C}=\overrightarrow{u}_{x}$ :
\begin{multline*}
	\left(\overrightarrow{v}\cdot\overrightarrow{\nabla}v_{x}\right)\overrightarrow{u}_{x} = \left(\overrightarrow{v}\cdot\overrightarrow{u}_{x}\right)\overrightarrow{\nabla}v_{x} - \overrightarrow{v}\wedge\left(\overrightarrow{\nabla}v_{x}\wedge\overrightarrow{u}_{x}\right)\\
	=v_{x}\overrightarrow{\nabla}v_{x}-\overrightarrow{v}\wedge\left(\overrightarrow{\nabla}v_{x}\wedge\overrightarrow{u}_{x}\right) = \frac{1}{2}\overrightarrow{\nabla}v_{x}^{2} - \overrightarrow{v}\wedge\left(\overrightarrow{\nabla}v_{x}\wedge\overrightarrow{u_{x}}\right)
\end{multline*}
Ainsi en procédant de la même façon pour les deux autres composantes, on obtient
	\[
	\left(\overrightarrow{v}\cdot\overrightarrow{\nabla}\right)\overrightarrow{v} =
	\frac{1}{2}\overrightarrow{\nabla}\left(v_{x}^{2}+v_{y}^{2}+v_{z}^{2}\right) - 
	\overrightarrow{v}\wedge\left(\overrightarrow{\nabla}v_{x}\wedge\overrightarrow{u}_{x}+\overrightarrow{\nabla}v_{y}\wedge\overrightarrow{u}_{y}+\overrightarrow{\nabla}v_{z}\wedge\overrightarrow{u}_{z}\right) 
	\]
On reconnait $v^2$ dans le gradient et l'on voit apparaître $\overrightarrow{\text{rot}}\overrightarrow{v}$ dans le dernier terme. On aboutit alors à une nouvelle expression de l'accélération
% --- equation --- (fold)
\begin{equation}
\fcolorbox{filet}{fond}{\hspace{0.5em}
\(\displaystyle 
\overrightarrow{a}(\text{M},t) = \dfrac{\partial \overrightarrow{v}}{\partial t} +	\overrightarrow{\text{grad}}\frac{v^{2}}{2}+\left(\overrightarrow{\text{rot}}\overrightarrow{v}\right)\wedge\overrightarrow{v}
\)\hspace{0.5em}}
\hspace{0.5em}\heartsuit
\end{equation}
%--- equation --- (end)

