\chapter*{Préface}
\addcontentsline{toc}{chapter}{Preface} % Add the preface to the table of contents as a chapter
La physique est une science qui repose essentiellement sur la mesure et la modélisation. Le bagage du physicien doit donc contenir les outils et les méthodes qui lui permettent :
\begin{enumerate}
	\item de tirer une information rationnelle à partir de ses mesures ;
	\item d'approcher le comportement d'un système modèle à l'aide d'une analyse mathématique.
\end{enumerate}
C'est pourquoi ce cours est découpé en deux parties : une première autour de la mesure, une autre autour des concepts mathématiques utiles au physicien.

Ce cours s'adresse à un public relativement large car il aborde différents outils mobilisés aussi bien en Lycée que dans l'Enseignement Supérieur.

\begin{flushright}
	\textit{Jimmy Roussel}
\end{flushright}
